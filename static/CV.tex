\documentclass[11pt, a4paper]{article}
\usepackage[margin=1cm]{geometry}
\usepackage{array}
\usepackage{longtable}
\usepackage{titlesec}
\usepackage{fontspec}
\usepackage{fancyhdr}
\usepackage{hyperref}
\usepackage{xcolor}
\definecolor{grey}{RGB}{153,153,153}
\hypersetup{
    colorlinks,
    linkcolor=grey,
    urlcolor=grey,
    pdftitle={CV Johan Renaudie},
    pdfauthor={Johan Renaudie},
    bookmarksopen=true,
    bookmarksopenlevel=1,
    pdfstartview=Fit
}

\setmainfont{Trade Gothic LT Std}

\titleformat{\section}{\large \bfseries}{\thesection}{1em}{}
\titleformat*{\subsection}{\normalsize \bfseries}
\setcounter{secnumdepth}{0}
\titlespacing{\section}{0pt}{3.25ex plus 1ex minus .2ex}{0pt}
\titlespacing{\subsection}{0pt}{0pt}{0pt}

\setlength{\LTpre}{0pt}
\newcommand{\dohang}{\hangindent1em\hangafter1 }
\newcolumntype{P}[1]{>{\everypar{\dohang}\dohang\raggedright\arraybackslash} p{#1}}

\pagestyle{fancy}
\fancyfoot{}
\fancyhead{}
\renewcommand{\headrulewidth}{0pt}

\begin{document}

\begin{longtable}{p{0.45\linewidth} >{\raggedleft\arraybackslash}p{0.5\linewidth}}
{\bfseries \Large Johan RENAUDIE} & born in Sarlat-la-Can\'{e}da (France), July the 26th, 1983 \\
\href{mailto:johan.renaudie@mfn.berlin}{johan.renaudie@mfn.berlin}  & ORCID: \href{http://orcid.org/0000-0002-9107-1984}{0000-0002-9107-1984}\\
\hline
\end{longtable}

\section{EDUCATION}
\begin{longtable}{p{0.1\linewidth} P{0.85\linewidth}}
2013 & Promotion (PhD) magna cum laude in Biology (Paleontology) at Humboldt-Universit\"{a}t, Berlin, Germany.\\
2007 & Master (MSc) in Systematic, Evolution and Paleontology at Universit\'{e} Pierre et Marie Curie (UPMC) coaccredited with Museum National d'Histoire Naturelle and \'{E}cole Normale Sup\'{e}rieure, Paris, France.\\
2004 & Licence (BSc) in Earth and Space Sciences at Universit\'{e} Paul Sabatier, Toulouse, France.\\
2003 & DEUG in Earth and Universe Sciences at Universit\'{e} Paul Sabatier, Toulouse, France.\\
2001 & Baccalaur\'{e}at in Science (Mathematics) at Lyc\'{e}e Pr\'{e} de Cordy, Sarlat-la-Can\'{e}da, France.\\
\end{longtable}

\section{RESEARCH EXPERIENCE}
\begin{longtable}{p{0.1\linewidth} P{0.85\linewidth}}
2018--22 & PostDoc research project (\href{https://www.daad.de/medien/hochschulen/regional/europa/mopga/projektbeschreibungen_englisch_15-05-2018.pdf}{DAAD MOPGA-GRI}) at the Museum f\"{u}r Naturkunde (MfN) with G. Asatryan and D. B. Lazarus on `Polar Oceans, Phytoplankton and Oceanic Carbon Sequestration in a Warm, High $p_{CO_2}$ World'.\\
2015--17 & PostDoc research project (\href{http://gepris.dfg.de/gepris/projekt/279867559}{DFG grant RE3470/3-1}) at MfN on `Diatoms, Radiolarians and the Cenozoic Silicon and Carbon cycles'.\\
%2015--17 & PostDoc research project (\href{http://gepris.dfg.de/gepris/projekt/279867559}{DFG grant RE3470/3-1}) at the Museum f\"{u}r Naturkunde (MfN) on `Diatoms, Radiolarians and the Cenozoic Silicon and Carbon cycles'.\\
2014--15 & PostDoc research project at MfN with D. B. Lazarus and H. P\"{a}like on `\href{http://earthtime-eu.eu/earthtime/?page_id=686}{Earthtime-EU}: Integrated deep-sea microfossil chronostratigraphic database, website and analytic tools'.\\
2008--12 & PhD research project (\href{http://gepris.dfg.de/gepris/projekt/84744046}{DFG grant LA1191/8-1,2}) at MfN with D. B. Lazarus and B. Mohr on a `Synthesis on Antarctic Neogene radiolarians: taxonomy, macroevolution and biostratigraphy'.\\
2007 & MSc research project at UPMC with T. Danelian and S. Saint-Martin on `Siliceous plankton paleoecology in the tropical Atlantic in relation with Middle Eocene climatic changes'.\\
2006 & MSc research project at UPMC with T. Danelian on `Radiolarian diversity and taphonomy during the critical warming interval of the Paleocene-Eocene boundary'.\\
\end{longtable}

\section{PUBLICATIONS}
\subsection{Peer-reviewed articles}
\begin{longtable}{p{0.1\linewidth} P{0.85\linewidth}}
2018 & Renaudie J., Drews E.-L., B\"{o}hne S. \href{http://dx.doi.org/10.5194/fr-21-183-2018}{The Paleocene record of marine diatoms in deep-sea sediments}. \emph{Fossil Record}, 21(2), 183--205.\\
2016 & Renaudie J. \href{http://dx.doi.org/10.5194/bg-13-6003-2016}{Quantifying the Cenozoic marine diatom deposition history: links to the C and Si cycles.} \emph{Biogeosciences}, 13(21), 6003-6014.\\
 & Wiese R., Renaudie J., Lazarus, D.B. \href{http://dx.doi.org/10.1130/G38347.1}{Testing the accuracy of genus-level data to predict species diversity in Cenozoic marine diatoms.} \emph{Geology}, 44(12), 1051-1054.\\
 & Renaudie J., Lazarus D.B. \href{http://dx.doi.org/10.1144/jmpaleo2014-026}{New species of Neogene radiolarians from the Southern Ocean - Part IV.} \emph{Journal of Micropalaeontology}, 35(1), 26-53.\\
2015 & Renaudie J., Lazarus D.B. \href{http://dx.doi.org/10.1144/jmpaleo2013-034}{New species of Neogene radiolarians from the Southern Ocean - Part III.} \emph{Journal of Micropalaeontology}, 34(2), 181-209.\\
2014 & Lazarus D.B., Barron J., Renaudie J., Diver P., Türke A. \href{http://dx.doi.org/10.1371/journal.pone.0084857}{Cenozoic diatom diversity and correlation to climate change.} \emph{PLoS ONE}, 9(1), e84857.\\
2013 & Renaudie J., Lazarus D.B. \href{http://dx.doi.org/10.1144/jmpaleo2011-025}{New species of Neogene radiolarians from the Southern Ocean - Part II.}\emph{ Journal of Micropalaeontology}, 32(1), 59-86.\\
 & Renaudie J., Lazarus D.B. \href{http://dx.doi.org/10.1666/12016}{On the accuracy of paleodiversity reconstructions: a case study in antarctic radiolarians.} \emph{Paleobiology}, 39(3), 491-509.\\
2012 & Renaudie J., Lazarus D.B. \href{http://dx.doi.org/10.1144/0262-821X10-026}{New species of Neogene radiolarians from the Southern Ocean.} \emph{Journal of Micropalaeontology}, 31(1), 29-52.\\
2010 & Renaudie J., Danelian T., Saint-Martin S., Le Callonec L., Tribovillard N. \href{http://dx.doi.org/10.1016/j.palaeo.2009.12.004}{Siliceous phytoplankton response to a Middle Eocene warming event recorded in the tropical Atlantic (Demerara Rise, ODP Site 1260A).} \emph{Palaeogeography, Palaeoclimatology, Palaeoecology}, 286, 121--134.\\
\end{longtable}

\subsection{Submitted}
\begin{longtable}{p{0.1\linewidth} P{0.85\linewidth}}
2018 & Lazarus D.B., Renaudie J., Lenz D., Diver P., Klump J. \href{https://peerj.com/preprints/26836}{Raritas and RaritasVox: programs for counting high diversity categorical data with highly unequal abundances}. \emph{Submitted to PeerJ}.\\
% & Lazarus D.B., Gray R., Renaudie J. Neural net classification of unedited transmitted light microscope images of radiolaria from a sparse dataset. \emph{Submitted to PeerJ}.\\
\end{longtable}

%\section{FIELDWORK}
%\begin{longtable}{p{0.1\linewidth} P{0.85\linewidth}}
%2019 & \href{https://iodp.tamu.edu/scienceops/expeditions/amundsen_sea_ice_sheet_history.html}{IODP Expedition 379 to the Amundsen Sea}: Palaeontology (radiolarians).\\
%\end{longtable}

\section{TALKS}
\subsection{Invited talks}
\begin{longtable}{p{0.1\linewidth} P{0.85\linewidth}}
2016 & University of Leeds, UK : Diatoms, climate and the marine Silicon cycle. November, 10th.\\
\end{longtable}
\subsection{Conference talks}
\begin{longtable}{p{0.1\linewidth} P{0.85\linewidth}}
2017 & InterRad 15th meeting in Niigata, Japan: Renaudie J., Fontorbe G., Lazarus D., Salzmann S., Frings P., Conley D., Testing the vital effect on silicon isotope measurements in Late Eocene Pacific radiolarians.\\
2016 & Lyell Meeting of the Geological Society in London, UK : Lazarus D., Renaudie J., Diver P., The NSB (Neptune) Database : current status and future development.\\
2015 & InterRad 14th meeting in Antalya, Turkey: Renaudie J., Lazarus D.B., Diatoms, radiolarians and the Cenozoic Si and C cycles ; Lazarus D.B., Renaudie J., Reconstructing radiolarian diversity: what we don't know; and a new analysis of the Cenozoic.\\
2014 & EGU general meeting in Vienna, Austria: Renaudie J., Lazarus D.B., A quantitative review of Cenozoic diatom deposition history.\\
2013 & TMS Silicofossil Group meeting in Cambridge, UK: Renaudie J., Lazarus D.B., Cenozoic history of marine diatom deposition: a quantitative review.\\
 & Deutscher Museumsbund Fachgruppe Dokumentation Herbsttagung in Berlin-Dahlem, Germany: Renaudie J. (on behalf of Lazarus D.B.), Taxonomic backbone databases for fossil and living marine microplankton.\\
2012 & InterRad 13th meeting in Cadiz, Spain: Renaudie J., Lazarus D.B., On the accuracy of paleodiversity reconstructions: a case study using Antarctic Neogene radiolarians.\\
 & Jubil\"{a}umstagung der Pal\"{a}ontologische Gesellschaft in Berlin, Germany: Renaudie J., Lazarus D.B., Macroevolutionary patterns in Antarctic Neogene radiolarians.\\
 & TMS annual meeting in Nottingham, UK: Renaudie J., Lazarus D.B., Toward an high-resolution stratigraphy for Antarctic Neogene radiolarians.\\
2011 & TMS Silicofossil Group meeting in Lille, France: Renaudie J., Lazarus D.B., Macro-evolutionary patterns in Antarctic Neogene radiolarians.\\
2007 & Congr\`{e}s de l'Association Fran\c{c}aise de Pal\'{e}ontologie in Dignes-les-Bains, France: Renaudie J., Danelian T., Saint-Martin S., Eocene diatoms in tropical Atlantic (ODP Leg 207) and climate changes.\\
\end{longtable}
\subsection{In-house talks}
\begin{longtable}{p{0.1\linewidth} P{0.85\linewidth}}
2017 & Wissenschaftstag `Taxonomie': Micropaleontology, between taxonomic backbone databases and alpha taxonomy. May, 31st.\\
\end{longtable}

\section{POSTERS}
\begin{longtable}{p{0.1\linewidth} P{0.85\linewidth}}
2018 & 25th International Diatom Symposium in Berlin, Germany: Renaudie J., Lazarus D., Macroevolutionary patterns in Cenozoic marine diatoms from deep-sea sediments, and their relationship with climate and marine geochemical cycles.\\
2017 & Kolloquium der DFG-Schwerpunkte ICDP/IODP in Braunschweig, Germany: Renaudie J., Fontorbe G., Drews E.-F., Böhne S., Lazarus D., Constraining the history of the Cenozoic marine silicon cycle with siliceous microfossils.\\
 & Evolution 2017 in Portland, Oregon: Renaudie J., Lazarus D., Diver P., The NSB (Neptune) marine microfossil occurrences database.\\
2016 & Kolloquium der DFG-Schwerpunkte ICDP/IODP in Heidelberg, Germany: Renaudie J., Siliceous microfossils and the Cenozoic marine carbon and silicon cycles ; Renaudie J., Diver P., Lazarus D., NSB and ADP: a new, expanded and improved software system for marine planktonic microfossil and geochronologic data; Wiese R., Renaudie J., Lazarus D., Can genera be used as proxies for species in studies of biodiversity-climate sensitivity? A test using Cenozoic marine diatoms.\\
 & International Conference on Paleoceanography in Utrecht, Netherlands: Renaudie J., Quantifying the Cenozoic marine diatom record; Renaudie J., Lazarus D., Diver P., Expanding the NSB database for paleoceanographical research.\\
2015 & Kolloquium der DFG-Schwerpunkte ICDP/IODP in Bonn, Germany: Renaudie J., Diatoms and the Cenozoic Si and C cycles.\\
 & GSA annual meeting in Baltimore, USA: Renaudie J., Diver P., Lazarus D., NSB: a new, expanded and improved database of marine planktonic microfossil data.\\
2014 & Kolloquium der DFG-Schwerpunkte ICDP/IODP in Erlangen, Germany: Renaudie J., Lazarus D.B., A quantitative review of Cenozoic diatom deposition history.\\
 & EGU general meeting in Vienna, Austria: Lazarus D.B., Renaudie J., Diversity history of Cenozoic marine siliceous plankton.\\
2013 & Kolloquium der DFG-Schwerpunkte ICDP/IODP in Freiberg, Germany: Renaudie J., Lazarus D.B., Toward an high-resolution stratigraphy for Antarctic Neogene radiolarians.\\
2012 & Kolloquium der DFG-Schwerpunkte ICDP/IODP in Kiel, Germany: Renaudie J., Lazarus D.B., Macroevolutionary patterns in Antarctic Neogene radiolarians.\\
 & InterRad 13th meeting in Cadiz, Spain: Renaudie J., Lazarus D.B., Advances in Antarctic Neogene radiolarian high-resolution stratigraphy; Lazarus D.B., Renaudie J., Taxonomic documentation of Southern Ocean Neogene radiolarians.\\
 & TMS annual meeting in Nottingham, UK: Renaudie J., Lazarus D.B., Macroevolutionary patterns in Antarctic Neogene radiolarians.\\
2011 & Kolloquium der DFG-Schwerpunkte ICDP/IODP in Münster, Germany: Renaudie J., Lazarus D.B., Paleodiversity reconstructions in Antarctic Neogene radiolarians\\
2010 & Kolloquium der DFG-Schwerpunkte ICDP/IODP in Frankfurt, Germany: Renaudie J., Lazarus D.B., Weinkauf M., Marine micropaleontology, round two: whole faunal surveys and their use in biostratigraphy and macroevolutionary research.\\
 & Third International Paleontological Congress in London, UK: Renaudie J., Lazarus D.B., Macroevolutionary patterns in Antarctic Neogene radiolarians.\\
2009 & Kolloquium der DFG-Schwerpunkte ICDP/IODP in Potsdam, Germany: Lazarus D.B., Renaudie J., Synthesis and analysis of Antarctic Neogene Radiolaria.\\
 & InterRad 12th meeting in Nanjing, China: Lazarus D.B., Renaudie J., Synthesis on Antarctic Neogene Radiolaria: preliminary results.\\
2007 & Paleontological Association 51st Annual Meeting in Uppsala, Sweden: Renaudie J., Danelian T., Saint-Martin S., The siliceous plankton response of the equatorial Atlantic to the Middle Eocene Climatic Optimum event (ODP Site 1260, Demerara Rise, off Surinam).\\
2006 & Paleontological Association 50th Annual Meeting in Sheffield, UK: Danelian T., Renaudie J., Blanc-Valleron M.-M., The radiolarian record of the equatorial Atlantic during the Paleocene/Eocene Thermal Maximum event.\\
\end{longtable}

\section{SCIENTIFIC PROGRAMMING}
\subsection{Softwares}
\begin{longtable}{p{0.1\linewidth} P{0.85\linewidth}}
2017 & \href{http://github.com/plannapus/NSB_ADP_wx/releases}{NSB\_ADP\_wx} -- Age-Depth plot maker in Python.\\
2016 & \href{http://github.com/plannapus/Raritas/releases}{Raritas} -- Micropaleontological counting software in Python.\\
\end{longtable}
\subsection{Packages}
\begin{longtable}{p{0.1\linewidth} P{0.85\linewidth}}
2014 & \href{http://github.com/plannapus/NSB}{NSBcompanion} -- R package to retrieve data from NSB database.\\
2012 & \href{http://github.com/plannapus/CONOP9companion}{CONOP9companion} -- R package to integrate software CONOP9 in a statistical workflow.\\
\end{longtable}
\subsection{Databases}
\begin{longtable}{p{0.1\linewidth} P{0.85\linewidth}}
since 2013 & Maintainer and developer of the \href{http://nsb-mfn-berlin.de/}{NSB database}, successor of the legacy Neptune database.\\
\end{longtable}

\section{WORKSHOP PARTICIPATIONS}
\begin{longtable}{p{0.1\linewidth} P{0.85\linewidth}}
2017 & \href{https://abcd.biowikifarm.net/wiki/Events:WorkshopEFG2017}{`Access to Geosciences: sharing and publishing data related to paleontological, mineralogical, and petrological objects using a common data standard'} workshop at MfN.\\
2010 & `Paleobiology Database Intensive Workshop in Analytical Paleobiology' at Macquarie University, Sydney, Australia.\\
\end{longtable}

\section{PUBLIC OUTREACH}
\begin{longtable}{p{0.1\linewidth} P{0.85\linewidth}}
2014 & `Tiefenzeit Geschichten der Zukunft: von Plankton, Muscheln und Klimawandel' booth at the MfN during the \href{http://www.langenachtderwissenschaften.de/}{Lange Nacht der Wissenschaften}.\\
2011 & Guided tour of the MfN Micropaleontology Collection during the \href{http://www.lange-nacht-der-museen.de/}{Lange Nacht der Museen}.\\
\end{longtable}

% \section{TEACHING EXPERIENCE}
% \subsection{Mentoring}
% \begin{longtable}{p{0.1\linewidth} P{0.85\linewidth}}
% 2017 & Robert Seeger (Student helper) \\
% 2016 & Effi-Laura Drews (BSc Praktikum) and Simon B\"{o}hne (BSc Praktikum) \\
% 2015 & Mandy Schmohl (Student helper) \\
% 2014 & Robert Wiese (BSc Praktikum) \\
% \end{longtable}

\section{GRANTS \& AWARDS}
\subsection{Grants}
\begin{longtable}{p{0.1\linewidth} P{0.85\linewidth}}
% 2018 & DAAD `Make Our Planet Great Again -- German research Initiative' grant, as co-I (PI: Gayane Asatryan).\\
2015 & DFG Grant RE3470/3--1: `Eigene Stelle' grant in the Priority Program 527 (IODP).\\
\end{longtable}
\subsection{Awards}
\begin{longtable}{p{0.1\linewidth} P{0.85\linewidth}}
2012 & TMS Student Prize for best Poster at InterRad 13th meeting in Cadiz, Spain.\\
\end{longtable}

\section{SERVICES TO PROFESSION}
\begin{longtable}{P{0.94\linewidth} p{0.01\linewidth}}
\href{http://publons.com/a/1291154/}{Reviews} for \emph{Proceedings of the Royal Society B}; \emph{Paleoceanography}; \emph{Paleobiology}; \emph{Palaeogeography, Palaeoclimatology, Palaeoecology}. & \\
Co-founder and current organizer of the MfN `Code Clinic' Programming Club. & \\
\end{longtable}

\section{WORK EXPERIENCE}
\begin{longtable}{p{0.2\linewidth} P{0.75\linewidth}}
09/2018--07/2022 & Postdoctoral researcher (DAAD MOPGA-GRI) at the MfN.\\
11/2015--05/2017 & Postdoctoral researcher (DFG) at the MfN.\\
03/2014--06/2015 & Postdoctoral researcher (Earthtime-EU) at the MfN.\\
05/2013--07/2013 & Programmer at the MfN for the Neptune Database web interface.\\
11/2008--09/2012 & Scientific assistant at the MfN.\\
%%07--08/2008 & Bartender at the brasserie `Le Jardin Gourmand', Castelnaud-la-Chapelle.\\
10/2007--07/2008 & Library assistant at the Biblioth\`{e}que Centrale of Universit\'{e} Paris VII.\\
%%07--08/2003--2007 & Bartender and cashier at the restaurant `Le Bouffon', Sarlat-la-Can\'{e}da.\\
\end{longtable}

\section{PROFESSIONAL MEMBERSHIP}
\begin{longtable}{p{0.94\linewidth} p{0.01\linewidth}}
European Geosciences Union (EGU) & \\
International Association of Radiolarian Paleontologists (InterRad) & \\
The Micropalaeontological Society (TMS) & \\
\end{longtable}

\section{OTHER}
\begin{longtable}{P{0.94\linewidth} p{0.01\linewidth}}
Fluent in French (native) and English, intermediate level in Latin and German. & \\
Advanced programming skills in R, python (including Django and wxPython) and postgreSQL (and other SQL derivatives), intermediate level in \LaTeX. Familiar with Unix shell and Apache Webserver. & \\
\end{longtable}

\end{document}
