\documentclass[11pt, a4paper]{article}
\usepackage[margin=0.8cm]{geometry}
\usepackage{array}
\usepackage{longtable}
\usepackage{titlesec}
\usepackage{fontspec}
\usepackage{fancyhdr}
\usepackage{hyperref}
\usepackage{xcolor}

\definecolor{grey}{RGB}{153,153,153}
\hypersetup{
    colorlinks,
    linkcolor=grey,
    urlcolor=grey,
    pdftitle={CV Johan Renaudie},
    pdfauthor={Johan Renaudie},
    bookmarksopen=true,
    bookmarksopenlevel=1,
    pdfstartview=Fit
}
\setmainfont{Trade Gothic LT Std}
\titleformat{\section}{\large \bfseries}{\thesection}{1em}{}
\titleformat*{\subsection}{\normalsize \bfseries}
\setcounter{secnumdepth}{0}
\titlespacing{\section}{0pt}{3.25ex plus 1ex minus .2ex}{0pt}
\titlespacing{\subsection}{0pt}{0pt}{0pt}
\setlength{\LTpre}{0pt}
\newcommand{\dohang}{\hangindent1em\hangafter1 } %Hanging indent
\newcolumntype{P}[1]{>{\everypar{\dohang}\dohang\raggedright\arraybackslash} p{#1}} %Hanging indent in table
\pagestyle{fancy} %All that just to say I don't want headers or footers...
\fancyfoot{}
\fancyhead{}
\renewcommand{\headrulewidth}{0pt}
\renewcommand{\footnoterule}{\hrule width \linewidth height 1pt}
\makeatletter
\newcommand\footnoteref[1]{\protected@xdef\@thefnmark{\ref{#1}}\@footnotemark}
\makeatother

\begin{document}

\begin{longtable}{p{0.45\linewidth} >{\raggedleft\arraybackslash}p{0.5\linewidth}}
{\bfseries \Large Johan RENAUDIE} & born in Sarlat-la-Can\'{e}da (France), July the 26th, 1983 \\
\href{mailto:johan.renaudie@mfn.berlin}{johan.renaudie@mfn.berlin} & ORCID: \href{http://orcid.org/0000-0002-9107-1984}{0000-0002-9107-1984}\\
\hline
\end{longtable}

\section{EDUCATION}
\begin{longtable}{p{0.1\linewidth} P{0.85\linewidth}}
2013 & Promotion (PhD) magna cum laude in Biology (Paleontology) at Humboldt-Universit\"{a}t, Berlin, Germany.\\
2007 & Master (MSc) in Systematic, Evolution and Paleontology at Universit\'{e} Pierre et Marie Curie (UPMC) coaccredited with Museum National d'Histoire Naturelle and \'{E}cole Normale Sup\'{e}rieure, Paris, France.\\
2004 & Licence (BSc) in Earth and Space Sciences at Universit\'{e} Paul Sabatier, Toulouse, France.\\
2003 & DEUG in Earth and Universe Sciences at Universit\'{e} Paul Sabatier, Toulouse, France.\\
2001 & Baccalaur\'{e}at in Science (Mathematics) at Lyc\'{e}e Pr\'{e} de Cordy, Sarlat-la-Can\'{e}da, France.\\
\end{longtable}

\section{RESEARCH EXPERIENCE}
\begin{longtable}{p{0.1\linewidth} P{0.85\linewidth}}
2018--22 & PostDoc research project (\href{https://www.daad.de/medien/hochschulen/regional/europa/mopga/projektbeschreibungen_englisch_15-05-2018.pdf}{DAAD MOPGA-GRI}) at the Museum f\"{u}r Naturkunde (MfN) with G. Asatryan and D. B. Lazarus on `Polar Oceans, Phytoplankton and Oceanic Carbon Sequestration in a Warm, High $p_{CO_2}$ World'.\\
2015--17 & PostDoc research project (\href{http://gepris.dfg.de/gepris/projekt/279867559}{DFG grant RE3470/3-1}) at MfN on `Diatoms, Radiolarians and the Cenozoic Silicon and Carbon cycles'.\\
2014--15 & PostDoc research project at MfN with D. B. Lazarus and H. P\"{a}like on `\href{http://earthtime-eu.eu/earthtime/?page_id=686}{Earthtime-EU}: Integrated deep-sea microfossil chronostratigraphic database, website and analytic tools'.\\
2008--12 & PhD research project (\href{http://gepris.dfg.de/gepris/projekt/84744046}{DFG grant LA1191/8-1,2}) at MfN with D. B. Lazarus and B. Mohr on a \href{https://doi.org/10.18452/16985}{`Synthesis on Antarctic Neogene radiolarians: taxonomy, macroevolution and biostratigraphy'}.\\
2007 & MSc research project at UPMC with T. Danelian and S. Saint-Martin on `Siliceous plankton paleoecology in the tropical Atlantic in relation with Middle Eocene climatic changes'.\\
2006 & MSc research project at UPMC with T. Danelian on `Radiolarian diversity and taphonomy during the critical warming interval of the Paleocene-Eocene boundary'.\\
\end{longtable}

\section{FIELDWORK}
\begin{longtable}{p{0.1\linewidth} P{0.85\linewidth}}
2019 & \href{https://iodp.tamu.edu/scienceops/expeditions/amundsen_sea_ice_sheet_history.html}{IODP Expedition 379 `Amundsen Sea West Antarctic ice-sheet history'}: Radiolarian specialist.\\
\end{longtable}

\section{PUBLICATIONS}
\subsection{Peer-reviewed articles}
\begin{longtable}{p{0.1\linewidth} P{0.85\linewidth}}
2019 & Piazza V., Duarte L.V., Renaudie J., Aberhan M. \href{http://doi.org/10.1017/pab.2019.11}{Reductions in body size of benthic macro- invertebrates as a precursor of the Early Toarcian (Early Jurassic) extinction event in the Lusitanian Basin, Portugal}. \textit{Paleobiology}, 45(2), 296--316.\\
2018 & Renaudie J., Drews E.-L., B\"{o}hne S. \href{http://dx.doi.org/10.5194/fr-21-183-2018}{The Paleocene record of marine diatoms in deep-sea sediments}. \textit{Fossil Record}, 21(2), 183--205.\\
& Lazarus D.B., Renaudie J., Lenz D., Diver P., Klump J. \href{http://dx.doi.org/10.7717/peerj.5453}{Raritas: a program for counting high diversity categorical data with highly unequal abundances}. \textit{PeerJ}, 6, e5453.\\
2016 & Renaudie J. \href{http://dx.doi.org/10.5194/bg-13-6003-2016}{Quantifying the Cenozoic marine diatom deposition history: links to the C and Si cycles.} \textit{Biogeosciences}, 13(21), 6003--6014.\\
 & Wiese R., Renaudie J., Lazarus, D.B. \href{http://dx.doi.org/10.1130/G38347.1}{Testing the accuracy of genus-level data to predict species diversity in Cenozoic marine diatoms.} \textit{Geology}, 44(12), 1051--1054.\\
 & Renaudie J., Lazarus D.B. \href{http://dx.doi.org/10.1144/jmpaleo2014-026}{New species of Neogene radiolarians from the Southern Ocean - Part IV.} \textit{Journal of Micropalaeontology}, 35(1), 26--53.\\
2015 & Renaudie J., Lazarus D.B. \href{http://dx.doi.org/10.1144/jmpaleo2013-034}{New species of Neogene radiolarians from the Southern Ocean - Part III.} \textit{Journal of Micropalaeontology}, 34(2), 181--209.\\
2014 & Lazarus D.B., Barron J., Renaudie J., Diver P., Türke A. \href{http://dx.doi.org/10.1371/journal.pone.0084857}{Cenozoic diatom diversity and correlation to climate change.} \textit{PLoS ONE}, 9(1), e84857.\\
2013 & Renaudie J., Lazarus D.B. \href{http://dx.doi.org/10.1144/jmpaleo2011-025}{New species of Neogene radiolarians from the Southern Ocean - Part II.} \textit{Journal of Micropalaeontology}, 32(1), 59--86.\\
 & Renaudie J., Lazarus D.B. \href{http://dx.doi.org/10.1666/12016}{On the accuracy of paleodiversity reconstructions: a case study in antarctic radiolarians.} \textit{Paleobiology}, 39(3), 491--509.\\
2012 & Renaudie J., Lazarus D.B. \href{http://dx.doi.org/10.1144/0262-821X10-026}{New species of Neogene radiolarians from the Southern Ocean.} \textit{Journal of Micropalaeontology}, 31(1), 29--52.\\
2010 & Renaudie J., Danelian T., Saint-Martin S., Le Callonec L., Tribovillard N. \href{http://dx.doi.org/10.1016/j.palaeo.2009.12.004}{Siliceous phytoplankton response to a Middle Eocene warming event recorded in the tropical Atlantic (Demerara Rise, ODP Site 1260A).} \textit{Palaeogeography, Palaeoclimatology, Palaeoecology}, 286, 121--134.\\
\end{longtable}

% \subsection{Book chapters}
% \begin{longtable}{p{0.1\linewidth} P{0.85\linewidth}}
% 2020 & Expedition 379 Scientists\footnoteref{exp379scientists}. \href{https://doi.org/10.14379/iodp.proc.379.101.2020}{Expedition 379 Summary}. \textit{In} Gohl K., Wellner J., Klaus A. and the Expedition 379 Scientists, \textit{Amundsen Sea West Antarctic Ice Sheet History}. \href{}{Proceedings of the International Ocean Discovery Program, 379}: College Station, TX (IODP).\\
% & Expedition 379 Scientists\footnoteref{exp379scientists}. \href{https://doi.org/10.14379/iodp.proc.379.102.2020}{Expedition 379 Methods}. \textit{In} Gohl K., Wellner J., Klaus A. and the Expedition 379 Scientists, \textit{Amundsen Sea West Antarctic Ice Sheet History}. \href{}{Proceedings of the International Ocean Discovery Program, 379}: College Station, TX (IODP).\\
% & Expedition 379 Scientists\footnoteref{exp379scientists}. \href{https://doi.org/10.14379/iodp.proc.379.103.2020}{Site U1532}. \textit{In} Gohl K., Wellner J., Klaus A. and the Expedition 379 Scientists, \textit{Amundsen Sea West Antarctic Ice Sheet History}. \href{}{Proceedings of the International Ocean Discovery Program, 379}: College Station, TX (IODP).\\
% & Expedition 379 Scientists\footnoteref{exp379scientists}. \href{https://doi.org/10.14379/iodp.proc.379.104.2020}{Site U1533}. \textit{In} Gohl K., Wellner J., Klaus A. and the Expedition 379 Scientists, \textit{Amundsen Sea West Antarctic Ice Sheet History}. \href{}{Proceedings of the International Ocean Discovery Program, 379}: College Station, TX (IODP).\\
% \end{longtable}

\subsection{Other publications}
\begin{longtable}{p{0.1\linewidth} P{0.85\linewidth}}
2019 %& Gohl K., Wellner J., Klaus A. and the Expedition 379 Scientists\footnote{\label{exp379scientists}Gohl K., Wellner J., Klaus A., Bauersachs T., Bohaty S.M., Courtillat M., Cowan E.A., Esteves M.S.R., De Lira Mota M.A., Fegyveresi J.M., Frederichs T.W., Gao L., Halberstadt A.R., Hillenbrand C.-D., Horikawa K., Iwai M., Kim J.-H., King T.M., Klages J.P., Passchier S., Penkrot M.L., Prebble J.G., Rahaman W., Reinardy B.T.I., Renaudie J., Robinson D.E., Scherer R.P., Siddoway C.S., Wu L., Yamane M.}. \href{http://publications.iodp.org/preliminary_report/379/index.html}{Expedition 379 Preliminary Report: Amundsen Sea West Antarctic Ice Sheet History}. \textit{International Ocean Discovery Program: Preliminary Reports}, 379:1--x.\\
  & Varela S., Sbrocco E.J., Tarroso P., Perez-Luque A.J., Renaudie J., Warnst\"{a}dt N., Fandos G., Foster W.J., Tietje M. \href{http://dx.doi.org/10.7818/ECOS.1707}{BioExtreme hackathon en el Museum f\"{u}r Naturkunde de Berlín, Alemania.} \textit{Ecosistemas}, 28(1):129.\\
\end{longtable}

\subsection{Submitted}
\begin{longtable}{p{0.1\linewidth} P{0.85\linewidth}}
2018 & Renaudie J., Gray R., Lazarus D.B. \href{https://peerj.com/preprints/27328/}{Accuracy of a neural net classification of closely-related species of microfossils from a sparse dataset of unedited images.} \textit{Submitted to PeerJ}.\\
\end{longtable}

\section{SEMINAR TALKS}
\subsection{Invited talks}
\begin{longtable}{p{0.1\linewidth} P{0.85\linewidth}}
%2019 & Museum f\"{u}r Naturkunde Magdeburg, Germany: \href{}{Antarktisches Mikroplankton und vergangene Klimawandel}. November, 6th.\\
2016 & University of Leeds, UK : \href{http://www.see.leeds.ac.uk/research/iag/geoscience-seminars/event/?SemID=534}{Diatoms, climate and the marine Silicon cycle}. November, 10th.\\
\end{longtable}
\subsection{In-house talks}
\begin{longtable}{p{0.1\linewidth} P{0.85\linewidth}}
2018 & Evolutionsbiologisches Seminar (MfN): The Cenozoic evolution of the diatom-climate system. September, 27th.\\
2017 & Wissenschaftstag `Taxonomie' (MfN): Micropaleontology, between taxonomic backbone databases and alpha taxonomy. May, 31st.\\
\end{longtable}

\section{CONGRESS PARTICIPATIONS}
\subsection{Organization}
\begin{longtable}{p{0.1\linewidth} P{0.85\linewidth}}
2019 & EGU general meeting in Vienna, Austria: \href{https://meetingorganizer.copernicus.org/EGU2019/session/31041}{`SSP4.6: Plankton in modern and past ecosystems'} (convener: Thibault N.; co-conveners: Bottini C., Luciani V., Renaudie J., Noble P.).
\end{longtable}
\subsection[Talks]{Talks \textnormal{\footnotesize{(\underline{speaker})}}}
\begin{longtable}{p{0.1\linewidth} P{0.85\linewidth}}
2019
% & 3rd International Congress on Stratigraphy in Milan, Italy: \underline{Renaudie J.}, Lazarus D., Diver P., NSB, a Big Data tool for chronostratigraphic syntheses of the deep-sea sediment record.\\
& North American Paleontological Conference (NAPC) in Riverside, USA: Renaudie J., \underline{Lazarus D.}, Asatryan G. Diversity dynamics and climate change in Cenozoic marine siliceous plankton; Renaudie J., Young J., \underline{Lazarus D.}, Diver P., NSB and Mikrotax: Databases and software tools for fossil and living plankton research; \underline{Trubovitz S.}, Lazarus D., Renaudie J., Noble P., Tropical and polar plankton demonstrate contrasting sensitivities to climate change throughout the Late Neogene.\\
% & Biodiversity\_Next in Leiden, Netherlands: \underline{Lazarus D.}, Renaudie J., Paleobiodiversity and Earth Science Environmental Data.\\
2018 & GEOBONN 2018 `Living Earth' in Bonn, Germany: \underline{Jansen M.}, Buchwitz M., Renaudie J., Voigt S., Reconstruction of an ancestral amniote trackmaker based on trackway data, track- trackmaker correlation and phylogeny.\\
2017 & InterRad 15th meeting in Niigata, Japan: \underline{Renaudie J.}, Fontorbe G., Lazarus D., Salzmann S., Frings P., Conley D., Testing the vital effect on silicon isotope measurements in Late Eocene Pacific radiolarians.\\
2016 & Lyell Meeting of the Geological Society in London, UK : \underline{Lazarus D.}, Renaudie J., Diver P., The NSB (Neptune) Database : current status and future development.\\
2015 & InterRad 14th meeting in Antalya, Turkey: \underline{Renaudie J.}, Lazarus D., Diatoms, radiolarians and the Cenozoic Si and C cycles ; \underline{Lazarus D.}, Renaudie J., Reconstructing radiolarian diversity: what we don't know; and a new analysis of the Cenozoic.\\
2014 & EGU general meeting in Vienna, Austria: \underline{Renaudie J.}, Lazarus D., A quantitative review of Cenozoic diatom deposition history.\\
2013 & TMS Silicofossil Group meeting in Cambridge, UK: \underline{Renaudie J.}, Lazarus D., Cenozoic history of marine diatom deposition: a quantitative review.\\
 & Deutscher Museumsbund Fachgruppe Dokumentation Herbsttagung in Berlin-Dahlem, Germany: \underline{Renaudie J.} (on behalf of Lazarus D.), Taxonomic backbone databases for fossil and living marine microplankton.\\
2012 & InterRad 13th meeting in Cadiz, Spain: \underline{Renaudie J.}, Lazarus D., On the accuracy of paleodiversity reconstructions: a case study using Antarctic Neogene radiolarians.\\
 & Jubil\"{a}umstagung der Pal\"{a}ontologische Gesellschaft in Berlin, Germany: \underline{Renaudie J.}, Lazarus D., Macroevolutionary patterns in Antarctic Neogene radiolarians.\\
 & TMS annual meeting in Nottingham, UK: \underline{Renaudie J.}, Lazarus D., Toward an high-resolution stratigraphy for Antarctic Neogene radiolarians.\\
2011 & TMS Silicofossil Group meeting in Lille, France: \underline{Renaudie J.}, Lazarus D., Macro-evolutionary patterns in Antarctic Neogene radiolarians.\\
2007 & Congr\`{e}s de l'Association Fran\c{c}aise de Pal\'{e}ontologie in Dignes-les-Bains, France: \underline{Renaudie J.}, Danelian T., Saint-Martin S., Eocene diatoms in the tropical Atlantic (ODP Leg 207) and climate changes.\\
\end{longtable}
\subsection[Posters]{Posters \textnormal{\footnotesize{(last 3 years only)}}}
\begin{longtable}{p{0.1\linewidth} P{0.85\linewidth}}
2019 & EGU general meeting in Vienna, Austria: Asatryan G., Lazarus D., Renaudie J., The preliminary studies of plankton in the framework of the project ``Paleogene Polar Plankton and Paleoproductivity''.\\
%  & Society of Vertebrate Paleontologists (SVP) Annual Meeting in Brisbane, Australia: Jansen M., Buchwitz M., Renaudie J., Voigt S., Reconstruction of an ancestral amniote trackmaker based on trackway data, trackmaker correlation and phylogeny.\\
2018 & TMS Annual meeting in Leeds, UK: Asatryan G., Lazarus D., Renaudie J., Paleogene polar phytoplankton and oceanic carbon sequestration; Renaudie J., Drews E.-L., Böhne S., The Paleocene fossil record of marine planktonic diatoms in deep-sea sediments; Renaudie J., Gray, R., Lazarus D., Testing the accuracy of the MobileNet convolutional neural network to identify closely-related radiolarian species based on a sparse dataset.\\
& 25th International Diatom Symposium in Berlin, Germany: Renaudie J., Lazarus D., Macroevolutionary patterns in Cenozoic marine diatoms from deep-sea sediments, and their relationship with climate and marine geochemical cycles.\\
2017 & Kolloquium der DFG-Schwerpunkte ICDP/IODP in Braunschweig, Germany: Renaudie J., Fontorbe G., Drews E.-F., Böhne S., Lazarus D., Constraining the history of the Cenozoic marine silicon cycle with siliceous microfossils.\\
 & Evolution 2017 in Portland, Oregon: Renaudie J., Lazarus D., Diver P., The NSB (Neptune) marine microfossil occurrences database.\\
% 2016 & Kolloquium der DFG-Schwerpunkte ICDP/IODP in Heidelberg, Germany: Renaudie J., Siliceous microfossils and the Cenozoic marine carbon and silicon cycles ; Renaudie J., Diver P., Lazarus D., NSB and ADP: a new, expanded and improved software system for marine planktonic microfossil and geochronologic data; Wiese R., Renaudie J., Lazarus D., Can genera be used as proxies for species in studies of biodiversity-climate sensitivity? A test using Cenozoic marine diatoms.\\
%  & International Conference on Paleoceanography in Utrecht, Netherlands: Renaudie J., Quantifying the Cenozoic marine diatom record; Renaudie J., Lazarus D., Diver P., Expanding the NSB database for paleoceanographical research.\\
%2015 & Kolloquium der DFG-Schwerpunkte ICDP/IODP in Bonn, Germany: Renaudie J., Diatoms and the Cenozoic Si and C cycles.\\
%  & GSA annual meeting in Baltimore, USA: Renaudie J., Diver P., Lazarus D., NSB: a new, expanded and improved database of marine planktonic microfossil data.\\
% 2014 %& Kolloquium der DFG-Schwerpunkte ICDP/IODP in Erlangen, Germany: Renaudie J., Lazarus D.B., A quantitative review of Cenozoic diatom deposition history.\\
%  & EGU general meeting in Vienna, Austria: Lazarus D.B., Renaudie J., Diversity history of Cenozoic marine siliceous plankton.\\
% 2013 & Kolloquium der DFG-Schwerpunkte ICDP/IODP in Freiberg, Germany: Renaudie J., Lazarus D.B., Toward an high-resolution stratigraphy for Antarctic Neogene radiolarians.\\
% 2012 %& Kolloquium der DFG-Schwerpunkte ICDP/IODP in Kiel, Germany: Renaudie J., Lazarus D.B., Macroevolutionary patterns in Antarctic Neogene radiolarians.\\
%  & InterRad 13th meeting in Cadiz, Spain: Renaudie J., Lazarus D.B., Advances in Antarctic Neogene radiolarian high-resolution stratigraphy; Lazarus D.B., Renaudie J., Taxonomic documentation of Southern Ocean Neogene radiolarians.\\
%  & TMS annual meeting in Nottingham, UK: Renaudie J., Lazarus D.B., Macroevolutionary patterns in Antarctic Neogene radiolarians.\\
% 2011 & Kolloquium der DFG-Schwerpunkte ICDP/IODP in Münster, Germany: Renaudie J., Lazarus D.B., Paleodiversity reconstructions in Antarctic Neogene radiolarians\\
% 2010 %& Kolloquium der DFG-Schwerpunkte ICDP/IODP in Frankfurt, Germany: Renaudie J., Lazarus D.B., Weinkauf M., Marine micropaleontology, round two: whole faunal surveys and their use in biostratigraphy and macroevolutionary research.\\
%  & Third International Paleontological Congress in London, UK: Renaudie J., Lazarus D.B., Macroevolutionary patterns in Antarctic Neogene radiolarians.\\
% 2009 & Kolloquium der DFG-Schwerpunkte ICDP/IODP in Potsdam, Germany: Lazarus D.B., Renaudie J., Synthesis and analysis of Antarctic Neogene Radiolaria.\\
%  & InterRad 12th meeting in Nanjing, China: Lazarus D.B., Renaudie J., Synthesis on Antarctic Neogene Radiolaria: preliminary results.\\
% 2007 & Paleontological Association 51st Annual Meeting in Uppsala, Sweden: Renaudie J., Danelian T., Saint-Martin S., The siliceous plankton response of the equatorial Atlantic to the Middle Eocene Climatic Optimum event (ODP Site 1260, Demerara Rise, off Surinam).\\
% 2006 & Paleontological Association 50th Annual Meeting in Sheffield, UK: Danelian T., Renaudie J., Blanc-Valleron M.-M., The radiolarian record of the equatorial Atlantic during the Paleocene/Eocene Thermal Maximum event.\\
since 2006 & 28 posters at 22 international conferences.
\end{longtable}
% \subsection{Interactive presentations (PICO)}
% \begin{longtable}{p{0.1\linewidth} P{0.85\linewidth}}
% \end{longtable}

\section{SCIENTIFIC PROGRAMMING}
\subsection{Softwares}
\begin{longtable}{p{0.1\linewidth} P{0.85\linewidth}}
2017 & \href{http://github.com/plannapus/NSB_ADP_wx/releases}{NSB\_ADP\_wx} -- Age-Depth plot maker in Python.\\
2016 & \href{http://github.com/plannapus/Raritas/releases}{Raritas} -- Micropaleontological counting software in Python.\\
\end{longtable}
\subsection{Packages}
\begin{longtable}{p{0.1\linewidth} P{0.85\linewidth}}
2014 & \href{http://github.com/plannapus/NSB}{NSBcompanion} -- R package to retrieve data from NSB database.\\
2013 & \href{http://github.com/plannapus/CONOP9companion}{CONOP9companion} -- R package to integrate software CONOP9 in a statistical workflow.\\
     & \href{https://cran.r-project.org/web/packages/dendextend/index.html}{dendextend} -- R package for dendrogram visualizations (\underline{as contributor only}).\\
\end{longtable}
\subsection{Databases}
\begin{longtable}{p{0.1\linewidth} P{0.85\linewidth}}
% 2019 & Co-creator of the ecoClimate's \href{}{BioExtreme} data package.
since 2013 & Maintainer and developer of the \href{http://nsb-mfn-berlin.de/}{NSB database}, successor of the legacy Neptune database.\\
\end{longtable}

\section{SERVICES TO PROFESSION}
\begin{longtable}{p{0.1\linewidth} P{0.85\linewidth}}
2018 & Remote reviewer for an ERC Advanced Grant proposal.\\
2015--18 & Organizer of the \href{https://github.com/plannapus/MfN-Code-Clinic}{MfN `Code Clinic' Scientific Programming Club} (on-the-job training for ECR). \\
since 2013 & \href{http://publons.com/a/1291154/}{Reviews} for \textit{Proceedings of the Royal Society B}; \textit{Paleoceanography}; \textit{Paleobiology}; \textit{Palaeogeography, Palaeoclimatology, Palaeoecology}; \textit{Journal of Plankton Research}; \textit{Water}; \textit{Revue de Micropal\'{e}ontologie}; \textit{Acta Palaeontologica Romaniae}. \\
\end{longtable}

\section{WORKSHOP PARTICIPATIONS}
\begin{longtable}{p{0.1\linewidth} P{0.85\linewidth}}
2018 & \href{https://github.com/macroecology/BioExtremes}{BioExtreme hackathon} at MfN.\\
2017 & \href{https://abcd.biowikifarm.net/wiki/Events:WorkshopEFG2017}{`Access to Geosciences: sharing and publishing data related to paleontological, mineralogical, and petrological objects using a common data standard'} workshop at MfN.\\
2010 & \href{http://fossilworks.org/?page=workshop}{`Paleobiology Database Intensive Workshop in Analytical Paleobiology'} at Macquarie University, Sydney, Australia.\\
\end{longtable}

\section{PUBLIC OUTREACH}
\begin{longtable}{p{0.1\linewidth} P{0.85\linewidth}}
%2019 %& Public talk on \href{}{``Antarktisches Mikroplankton und vergangene Klimawandel''} at the Museum f\"{u}r Naturkunde Magdeburg, Germany.\\
% & `The ocean’s plankton and climate change; how to see the future from the bottom of the ocean' booth at the MfN during the \href{http://www.langenachtderwissenschaften.de/}{Lange Nacht der Wissenschaften}.\\
2014 & `Tiefenzeit Geschichten der Zukunft: von Plankton, Muscheln und Klimawandel' booth at the MfN during the \href{http://www.langenachtderwissenschaften.de/}{Lange Nacht der Wissenschaften}.\\
2011 & Guided tour of the MfN Micropaleontology Collection during the \href{http://www.lange-nacht-der-museen.de/}{Lange Nacht der Museen}.\\
\end{longtable}

\section{GRANTS \& AWARDS}
\subsection{Grants}
\begin{longtable}{p{0.1\linewidth} P{0.85\linewidth}}
2018 & DAAD `Make Our Planet Great Again -- German Research Initiative' grant (\underline{PI: Gayane Asatryan}).\\
2015 & DFG Grant RE3470/3--1: `Eigene Stelle' grant in the Priority Program 527 (IODP).\\
\end{longtable}
\subsection{Awards}
\begin{longtable}{p{0.1\linewidth} P{0.85\linewidth}}
2012 & TMS Student Prize for best Poster at InterRad 13th meeting in Cadiz, Spain.\\
\end{longtable}

\section[MEDIA COVERAGE]{MEDIA COVERAGE \textnormal{\footnotesize{\\of projects I'm involved with (selection)}}}
\begin{longtable}{p{0.1\linewidth} P{0.85\linewidth}}
2019 & \href{http://dx.doi.org/10.1126/science.aax7040}{Newly drilled sediment cores could reveal how fast the Antarctic ice sheet will melt}, \emph{Science news}. April, 15th. [Expedition 379]\\
  & \href{https://www.bbc.com/news/science-environment-47711600}{Climate change: Drilling in `Iceberg Alley'}, \emph{BBC}. March, 27th. [Expedition 379]\\
 % & \href{https://www.museumfuernaturkunde.berlin/en/uber-uns/neuigkeiten/scientist-mfn-board-iodp-expedition-379-amundsen-sea-west-antarctic-research}{Scientist from the MfN on Board: IODP Expedition 379 in the Amundsen Sea (West Antarctic) on the Research Drill Ship JOIDES Resolution}, \emph{MfN website}. February, 27th [Expedition 379].\\
2018 & \href{https://www.daad.de/der-daad/daad-aktuell/de/66800-make-our-planet-great-again-german-research-initiative-forschung-fuer-die-zukunft-der-erde/}{``Make Our Planet Great Again -- German Research Initiative'': Forschung für die Zukunft der Erde}, \emph{DAAD Aktuell}. October, 11th. [MOPGA project]\\
\end{longtable}

\section{WORK EXPERIENCE}
\begin{longtable}{p{0.2\linewidth} P{0.75\linewidth}}
09/2018--06/2022 & Postdoctoral researcher (DAAD MOPGA-GRI) at the MfN.\\
11/2015--05/2017 & Postdoctoral researcher (DFG) at the MfN.\\
03/2014--06/2015 & Postdoctoral researcher (Earthtime-EU) at the MfN.\\
05--07/2013 & Programmer at the MfN for the Neptune Database web interface.\\
11/2008--09/2012 & Scientific assistant at the MfN.\\
07--08/2008 & Bartender at the brasserie `Le Jardin Gourmand', Castelnaud-la-Chapelle.\\
10/2007--07/2008 & Library assistant at the Biblioth\`{e}que Centrale of Universit\'{e} Paris VII.\\
07--08/2003--2007 & Bartender at the restaurant `Le Bouffon', Sarlat-la-Can\'{e}da.\\
\end{longtable}

\section{PROFESSIONAL MEMBERSHIP}
\begin{longtable}{p{0.94\linewidth} p{0.01\linewidth}}
European Geosciences Union (EGU) & \\
International Association of Radiolarian Scientists (InterRad) & \\
The Micropalaeontological Society (TMS) & \\
\end{longtable}

\section{OTHER}
\begin{longtable}{P{0.94\linewidth} p{0.01\linewidth}}
Fluent in French (native) and English, intermediate level in Latin and German. & \\
Advanced programming skills in R, python (including Django and wxPython), postgreSQL (and other SQL derivatives) and \LaTeX. Familiar with Unix shell and Apache Webserver. & \\
\end{longtable}

{\let\thefootnote\relax\footnote{Last updated on \today}}

\end{document}
