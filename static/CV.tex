\documentclass[11pt, a4paper]{article}
\usepackage[margin=0.65cm]{geometry}
\usepackage{array}
\usepackage{longtable,needspace}
\usepackage{titlesec}
\usepackage{fontspec}
\usepackage{fancyhdr}
\usepackage{hyperref}
\usepackage{xcolor}
\usepackage{endnotes}
\usepackage{hyperendnotes}

\definecolor{grey}{RGB}{153,153,153}
\hypersetup{
    colorlinks,
    linkcolor=grey,
    urlcolor=grey,
    pdftitle={CV Johan Renaudie},
    pdfauthor={Johan Renaudie},
    bookmarksopen=true,
    bookmarksopenlevel=1,
    pdfstartview=Fit
}
\setmainfont{Trade Gothic LT Std}
\titleformat{\section}{\large \bfseries}{\thesection}{1em}{}
\titleformat*{\subsection}{\normalsize \bfseries}
\setcounter{secnumdepth}{0}
\titlespacing{\section}{0pt}{3.25ex plus 1ex minus .2ex}{0pt}
\titlespacing{\subsection}{0pt}{0pt}{0pt}
\setlength{\LTpre}{0pt}
\newcommand{\dohang}{\hangindent1em\hangafter1 }
\newcolumntype{P}[1]{>{\everypar{\dohang}\dohang\raggedright\arraybackslash} p{#1}}
\pagestyle{fancy}
\fancyfoot{}
\fancyhead{}
\renewcommand{\headrulewidth}{0pt}
\renewcommand{\footnoterule}{\hrule width \linewidth height 1pt}
\makeatletter
\newcommand\fnoteref[1]{\protected@xdef\@theenmark{\ref{#1}}\@endnotemark}
\makeatother
\renewcommand*{\enoteheading}{}

\begin{document}
\begin{longtable}{@{}p{0.45\linewidth} >{\raggedleft\arraybackslash}p{0.5\linewidth}@{}}
{\bfseries \Large Johan RENAUDIE} & born in Sarlat-la-Can\'{e}da (France), July the 26th, 1983 \\
\href{mailto:johan.renaudie@mfn.berlin}{johan.renaudie@mfn.berlin} & ORCID: \href{http://orcid.org/0000-0002-9107-1984}{0000-0002-9107-1984}\\
\hline
\end{longtable}

\section{EDUCATION}
\begin{longtable}{@{}p{0.1\linewidth} P{0.85\linewidth}@{}}
2013 & Promotion (PhD) in Biology (Paleontology) at Humboldt-Universit\"{a}t, Berlin, Germany.\\
2007 & Master (MSc) in Systematic, Evolution and Paleontology at Universit\'{e} Pierre et Marie Curie (UPMC) coaccredited with Museum National d'Histoire Naturelle and \'{E}cole Normale Sup\'{e}rieure, Paris, France.\\
2004 & Licence (BSc) in Earth and Space Sciences at Universit\'{e} Paul Sabatier, Toulouse, France.\\
2003 & DEUG in Earth and Universe Sciences at Universit\'{e} Paul Sabatier, Toulouse, France.\\
2001 & Baccalaur\'{e}at in Science (Mathematics) at Lyc\'{e}e Pr\'{e} de Cordy, Sarlat-la-Can\'{e}da, France.\\
\end{longtable}

\section{RESEARCH EXPERIENCE}
\begin{longtable}{@{}p{0.1\linewidth} P{0.85\linewidth}@{}}
2018--22 & PostDoc research project (\href{https://www.daad.de/medien/hochschulen/regional/europa/mopga/projektbeschreibungen_englisch_15-05-2018.pdf}{DAAD MOPGA-GRI grant 57429681}) at the Museum f\"{u}r Naturkunde (MfN) with G. Asatryan and D.B. Lazarus on `Polar Paleogene Plankton and Productivity'.\\
2015--17 & PostDoc research project (\href{http://gepris.dfg.de/gepris/projekt/279867559}{DFG grant RE3470/3-1}) at MfN on `Diatoms, Radiolarians and the Cenozoic Silicon and Carbon cycles'.\\
2014--15 & PostDoc research project at MfN with D.B. Lazarus and H. P\"{a}like on `\href{http://earthtime-eu.eu/earthtime/?page_id=686}{Earthtime-EU}: Integrated deep-sea microfossil chronostratigraphic database, website and analytic tools'.\\
2008--12 & PhD research project (\href{http://gepris.dfg.de/gepris/projekt/84744046}{DFG grant LA1191/8-1,2}) at MfN with D.B. Lazarus and B. Mohr on a \href{https://doi.org/10.18452/16985}{`Synthesis on Antarctic Neogene radiolarians: taxonomy, macroevolution and biostratigraphy'}.\\
2007 & MSc research project at UPMC with T. Danelian and S. Saint-Martin on `Siliceous plankton paleoecology in the tropical Atlantic in relation with Middle Eocene climatic changes'.\\
2006 & MSc research project at UPMC with T. Danelian on `Radiolarian diversity and taphonomy during the critical warming interval of the Paleocene-Eocene boundary'.\\
\end{longtable}

\section{FIELDWORK PARTICIPATIONS}
\begin{longtable}{@{}p{0.1\linewidth} P{0.85\linewidth}@{}}
% 2021 & \href{https://iodp.tamu.edu/scienceops/expeditions/norwegian_continental_margin_magmatism.html}{IODP Leg 396 `Mid-Norwegian continental margin magmatism'}: Shore-based radiolarian specialist.\\
2019 & \href{https://iodp.tamu.edu/scienceops/expeditions/amundsen_sea_ice_sheet_history.html}{IODP Leg 379 `Amundsen Sea West Antarctic ice-sheet history'}: Shipboard radiolarian specialist.\\
\end{longtable}

\section{PUBLICATIONS}
\subsection{Peer-reviewed articles}
\begin{longtable}{@{}p{0.1\linewidth} P{0.85\linewidth}@{}}
2021
    & Fenton I., Woodhouse A., Aze T., Lazarus D., Renaudie J., Dunhill A., Young J., Saupe E. \href{https://www.nature.com/articles/s41597-021-00942-7}{Triton, a new species-level database of Cenozoic planktonic foraminiferal occurrences}. \textit{Scientific Data}, 8:160.\\
    & Buchwitz M., Jansen M.A., Renaudie J., Marchetti L., Voigt S. \href{http://doi.org/10.3389/fevo.2021.674779}{Evolutionary change in locomotion close to the origin of amniotes inferred in a phylogenetically informed analysis of trackway data}. \textit{Frontiers in Ecology and Evolution}, 9:674779.\\
    & Gohl K., Uenzelmann-Neben G., Gille-Petzoldt J., Hillenbrand C.-D., Klages J.P., Bohaty S.M., Passchier S., Frederichs T., Wellner J.S., Lamb R., Leitchenkov G., IODP Expedition 379 Scientists\endnote{\label{exp379scientists2}Klaus A., Kulhanek D., Bauersachs T., Courtillat M., Cowan E.A., Esteves M.S.R., De Lira Mota M.A., Fegyveresi J.M., Gao L., Halberstadt A.R., Horikawa K., Iwai M., Kim J.-H., King T.M., Penkrot M.L., Prebble J.G., Rahaman W., Reinardy B.T.I., Renaudie J., Robinson D.E., Scherer R.P., Siddoway C.S., Wu L., Yamane M.}. \href{https://agupubs.onlinelibrary.wiley.com/doi/10.1029/2021GL093103?af=R}{Evidence for a highly dynamic West Antarctic Ice Sheet during the Pliocene}. \textit{Geophysical Research Letters}, 48:e2021GL093103.\\
2020 & Trubovitz S., Lazarus D., Renaudie J., Noble P. \href{http://doi.org/10.1038/s41467-020-18879-7}{Marine plankton show threshold extinction response to Neogene climate change}. \textit{Nature Communications}, 11:5069.\\
    & Renaudie J., Lazarus D.B., Diver P. \href{https://palaeo-electronica.org/content/2020/2966-the-nsb-database}{NSB (Neptune Sandbox Berlin): An expanded and improved database of marine planktonic microfossil data and deep-sea stratigraphy.}. \textit{Palaeontologia Electronica}, 23(2):a11.\\
2019 & Piazza V., Duarte L.V., Renaudie J., Aberhan M. \href{http://doi.org/10.1017/pab.2019.11}{Reductions in body size of benthic macro- invertebrates as a precursor of the Early Toarcian (Early Jurassic) extinction event in the Lusitanian Basin, Portugal}. \textit{Paleobiology}, 45(2), 296--316.\\
2018 & Renaudie J., Drews E.-L., B\"{o}hne S. \href{http://dx.doi.org/10.5194/fr-21-183-2018}{The Paleocene record of marine diatoms in deep-sea sediments}. \textit{Fossil Record}, 21(2), 183--205.\\
    & Lazarus D.B., Renaudie J., Lenz D., Diver P., Klump J. \href{http://dx.doi.org/10.7717/peerj.5453}{Raritas: a program for counting high diversity categorical data with highly unequal abundances}. \textit{PeerJ}, 6, e5453.\\
2016 & Renaudie J. \href{http://dx.doi.org/10.5194/bg-13-6003-2016}{Quantifying the Cenozoic marine diatom deposition history: links to the C and Si cycles.} \textit{Biogeosciences}, 13(21), 6003--6014.\\
    & Wiese R., Renaudie J., Lazarus, D.B. \href{http://dx.doi.org/10.1130/G38347.1}{Testing the accuracy of genus-level data to predict species diversity in Cenozoic marine diatoms.} \textit{Geology}, 44(12), 1051--1054.\\
    & Renaudie J., Lazarus D.B. \href{http://dx.doi.org/10.1144/jmpaleo2014-026}{New species of Neogene radiolarians from the Southern Ocean - Part IV.} \textit{Journal of Micropalaeontology}, 35(1), 26--53.\\
2015 & Renaudie J., Lazarus D.B. \href{http://dx.doi.org/10.1144/jmpaleo2013-034}{New species of Neogene radiolarians from the Southern Ocean - Part III.} \textit{Journal of Micropalaeontology}, 34(2), 181--209.\\
2014 & Lazarus D.B., Barron J., Renaudie J., Diver P., Türke A. \href{http://dx.doi.org/10.1371/journal.pone.0084857}{Cenozoic diatom diversity and correlation to climate change.} \textit{PLoS ONE}, 9(1), e84857.\\
2013 & Renaudie J., Lazarus D.B. \href{http://dx.doi.org/10.1144/jmpaleo2011-025}{New species of Neogene radiolarians from the Southern Ocean - Part II.} \textit{Journal of Micropalaeontology}, 32(1), 59--86.\\
    & Renaudie J., Lazarus D.B. \href{http://dx.doi.org/10.1666/12016}{On the accuracy of paleodiversity reconstructions: a case study in antarctic radiolarians.} \textit{Paleobiology}, 39(3), 491--509.\\
2012 & Renaudie J., Lazarus D.B. \href{http://dx.doi.org/10.1144/0262-821X10-026}{New species of Neogene radiolarians from the Southern Ocean.} \textit{Journal of Micropalaeontology}, 31(1), 29--52.\\
2010 & Renaudie J., Danelian T., Saint-Martin S., Le Callonec L., Tribovillard N. \href{http://dx.doi.org/10.1016/j.palaeo.2009.12.004}{Siliceous phytoplankton response to a Middle Eocene warming event recorded in the tropical Atlantic (Demerara Rise, ODP Site 1260A).} \textit{Palaeogeography, Palaeoclimatology, Palaeoecology}, 286, 121--134.\\
\end{longtable}

\subsection{Other publications}
\begin{longtable}{@{}p{0.1\linewidth} P{0.85\linewidth}@{}}
2021 
    & Gohl K., Wellner J., Klaus A. and the Expedition 379 Scientists\fnoteref{exp379scientists}. \href{https://doi.org/10.14379/iodp.proc.379.101.2021}{Expedition 379 Summary}. \textit{In} Gohl K., Wellner J., Klaus A. and the Expedition 379 Scientists, \textit{Amundsen Sea West Antarctic Ice Sheet History}. \href{http://publications.iodp.org/proceedings/379/379title.html}{Proceedings of the International Ocean Discovery Program, 379}: 1--21.\\
    & Gohl K., Wellner J., Klaus A. and the Expedition 379 Scientists\endnote{\label{exp379scientists}Gohl K., Wellner J., Klaus A., Bauersachs T., Bohaty S.M., Courtillat M., Cowan E.A., Esteves M.S.R., De Lira Mota M.A., Fegyveresi J.M., Frederichs T.W., Gao L., Halberstadt A.R., Hillenbrand C.-D., Horikawa K., Iwai M., Kim J.-H., King T.M., Klages J.P., Passchier S., Penkrot M.L., Prebble J.G., Rahaman W., Reinardy B.T.I., Renaudie J., Robinson D.E., Scherer R.P., Siddoway C.S., Wu L., Yamane M.}. \href{https://doi.org/10.14379/iodp.proc.379.102.2021}{Expedition 379 Methods}. \textit{In} Gohl K., Wellner J., Klaus A. and the Expedition 379 Scientists, \textit{Amundsen Sea West Antarctic Ice Sheet History}. \href{http://publications.iodp.org/proceedings/379/379title.html}{Proceedings of the International Ocean Discovery Program, 379}: 1--42.\\
    & Wellner J., Gohl K., Klaus A. and the Expedition 379 Scientists\fnoteref{exp379scientists}. \href{https://doi.org/10.14379/iodp.proc.379.103.2021}{Site U1532}. \textit{In} Gohl K., Wellner J., Klaus A. and the Expedition 379 Scientists, \textit{Amundsen Sea West Antarctic Ice Sheet History}. \href{http://publications.iodp.org/proceedings/379/379title.html}{Proceedings of the International Ocean Discovery Program, 379}: 1--47.\\
    & Wellner J., Gohl K., Klaus A. and the Expedition 379 Scientists\fnoteref{exp379scientists}. \href{https://doi.org/10.14379/iodp.proc.379.104.2021}{Site U1533}. \textit{In} Gohl K., Wellner J., Klaus A. and the Expedition 379 Scientists, \textit{Amundsen Sea West Antarctic Ice Sheet History}. \href{http://publications.iodp.org/proceedings/379/379title.html}{Proceedings of the International Ocean Discovery Program, 379}: 1--46.\\
2019 & Gohl K., Wellner J., Klaus A. and the Expedition 379 Scientists\fnoteref{exp379scientists}. \href{http://publications.iodp.org/preliminary_report/379/index.html}{Expedition 379 Preliminary Report: Amundsen Sea West Antarctic Ice Sheet History}. \textit{International Ocean Discovery Program: Preliminary Reports}, 379:1--33.\\
    & Varela S., Sbrocco E.J., Tarroso P., Perez-Luque A.J., Renaudie J., Warnst\"{a}dt N., Fandos G., Foster W.J., Tietje M. \href{http://dx.doi.org/10.7818/ECOS.1707}{BioExtreme hackathon en el Museum f\"{u}r Naturkunde de Berlín, Alemania.} \textit{Ecosistemas}, 28(1):129.\\
\end{longtable}

\subsection{Preprints}
\begin{longtable}{@{}p{0.1\linewidth} P{0.85\linewidth}@{}}
2018 & Renaudie J., Gray R., Lazarus D.B. \href{https://peerj.com/preprints/27328/}{Accuracy of a neural net classification of closely-related species of microfossils from a sparse dataset of unedited images.} \textit{Submitted to PeerJ}.\\
\end{longtable}

\section{SEMINAR TALKS}
\subsection{Invited talks}
\begin{longtable}{@{}p{0.1\linewidth} P{0.85\linewidth}@{}}
2019 & Museum f\"{u}r Naturkunde Magdeburg, Germany: Antarktisches Mikroplankton und vergangene Klimawandel. November, 6th.\\
    & GFZ-Potsdam, Germany: \href{http://plannapus.github.io/data/20190723Potsdam.pdf}{Cenozoic changes in the Si and C marine cycles from the point of view of diatoms}. July, 23rd.\\
2016 & University of Leeds, UK : Diatoms, climate and the marine Silicon cycle. November, 10th.\\
\end{longtable}
\subsection{In-house talks}
\begin{longtable}{@{}p{0.1\linewidth} P{0.85\linewidth}@{}}
2020 & \iffalse Wissenschaftstag `Lighthouse Projects' (MfN): \fi Automatic species counting for biodiversity and climate change research using AI and massive collection imaging. September, 21st.\\
2018 & \iffalse Evolutionsbiologisches Seminar (MfN): \fi The Cenozoic evolution of the diatom-climate system. September, 27th.\\
2017 & \iffalse Wissenschaftstag `Taxonomie' (MfN): \fi Micropaleontology, between taxonomic backbone databases and alpha taxonomy. May, 31st.\\
\end{longtable}

\section{CONGRESS PARTICIPATIONS}
\subsection{Organization}
\begin{longtable}{@{}p{0.1\linewidth} P{0.85\linewidth}@{}}
2019 & EGU general meeting in Vienna, Austria: \href{https://meetingorganizer.copernicus.org/EGU2019/session/31041}{`SSP4.6: Plankton in modern and past ecosystems'} (convener: Thibault N.; co-conveners: Bottini C., Luciani V., Renaudie J., Noble P.).
\end{longtable}
\Needspace{6\baselineskip}
\subsection{Displays}
\begin{longtable}{@{}p{0.1\linewidth} P{0.85\linewidth}@{}}
2020 & EGU2020 Sharing Geoscience Online: \underline{Renaudie J.}, Lazarus D., Trubovitz S., \"{O}zen V., Rodrigues de Faria G., Asatryan G., Noble P., \href{https://meetingorganizer.copernicus.org/EGU2020/EGU2020-3456.html}{Cenozoic plankton diversity dynamics and the impact of macroevolution on the marine carbon cycle}; \underline{Rodrigues de Faria G.}, Lazarus D., Struck U., Asatryan G., Renaudie J., \"{O}zen V., \href{https://meetingorganizer.copernicus.org/EGU2020/EGU2020-5924.html}{Paleogene Polar Plankton and export productivity changes between the Eocene and Oligocene}.\\
\end{longtable}
\subsection[Talks]{Talks \textnormal{\footnotesize{(last as speaker only)}}}
\begin{longtable}{@{}p{0.1\linewidth} P{0.85\linewidth}@{}}
2021 & Crossing the Paleontological-Ecological Gap online conference: Renaudie J., Özen V., Rodrigues de Faria G., Trubovitz S., Lazarus D., Climatic range of modern fossilizable phytoplankton.\\
2007--now  & \href{http://plannapus.github.io/static/conffull.pdf}{32 talks (including 12 as speaker) at 24 international conferences}.
\end{longtable}
\subsection[Posters]{Posters \textnormal{\footnotesize{(\underline{presenting}; last year only)}}}
\begin{longtable}{@{}p{0.1\linewidth} P{0.85\linewidth}@{}}
2020 & Ocean Sciences meeting in San Diego, USA: \underline{Trubovitz S.}, Lazarus D., Renaudie J., Noble P., Radiolarians exhibit a threshold response to climate change during the late Neogene.\\
 & TMS General Meeting online: \underline{\"{O}zen V.}, Rodrigues de Faria G., Renaudie J., Lazarus D., Southern Ocean diatom diversity at the Eocene-Oligocene transition; \underline{Rodrigues de Faria G.}, Renaudie J., Struck U., Lazarus D., Southern Ocean productivity across the Eocene-Oligocene boundary.\\
2006--now  & \href{http://plannapus.github.io/static/conffull.pdf}{34 posters (including 17 as presenting author) at 27 international conferences}.
\end{longtable}

\section{SCIENTIFIC PROGRAMMING}
\subsection{Softwares}
\begin{longtable}{@{}p{0.1\linewidth} P{0.85\linewidth}@{}}
2021 & \href{http://github.com/plannapus/raupShiny/releases}{raupShiny} -- Shiny App to display Raup's coiling model (last update: \href{https://doi.org/10.5281/zenodo.5171827}{version 1.1}; 2021).\\
2017 & \href{http://github.com/plannapus/NSB_ADP_wx/releases}{NSB\_ADP\_wx} -- Age-Depth plot maker in Python (last update: \href{http://doi.org/10.5281/zenodo.3408657}{version 0.7}; 2019).\\
2016 & \href{http://github.com/plannapus/Raritas/releases}{Raritas} -- Micropaleontological counting software in Python (last update: \href{https://github.com/plannapus/Raritas/releases/tag/v0.7}{version 0.7}; 2018).\\
\end{longtable}
\subsection{Packages}
\begin{longtable}{@{}p{0.1\linewidth} P{0.85\linewidth}@{}}
2014 & \href{http://github.com/plannapus/NSBcompanion}{NSBcompanion} -- R package to work with the NSB database (last update: \href{http://doi.org/10.5281/zenodo.3408198}{version 2.1}; 2019).\\
2013 & \href{http://github.com/plannapus/CONOP9companion}{CONOP9companion} -- R package to integrate software CONOP9 in a statistical workflow.\\
     & \href{https://cran.r-project.org/web/packages/dendextend/index.html}{dendextend} -- R package for dendrogram visualizations (\underline{as contributor only}).\\
\end{longtable}
\subsection{Databases}
\begin{longtable}{@{}p{0.1\linewidth} P{0.85\linewidth}@{}}
2013--now  & Maintainer and developer of the \href{http://nsb-mfn-berlin.de/}{NSB database}, successor of the legacy Neptune database.\\
\end{longtable}

\section{SERVICES TO PROFESSION}
\begin{longtable}{@{}p{0.1\linewidth} P{0.85\linewidth}@{}}
%2022 & Scientific board member for INTERRAD meeting.
2021 & Scientific expert for the evaluation process of the ANR Generic Call.\\
2021--now & Reviewer board member for \href{https://www.mdpi.com/journal/biology/submission_reviewers}{Biology}.\\
2019 & Outside reader for an MSc defense (William Bugbee; University of Northern Illinois, USA).\\
2018 & Remote reviewer for an ERC Advanced Grant proposal.\\
2015--18 & Organizer of the \href{https://github.com/plannapus/MfN-Code-Clinic}{MfN `Code Clinic' Scientific Programming Club} (on-the-job training for ECR). \\
2013--now  & \href{http://publons.com/a/1291154/}{Reviews} for \textit{Proceedings of the Royal Society B}; \textit{Paleoceanography}; \textit{Paleobiology}; \textit{Palaeo\-geography, Palaeo\-climatology, Palaeo\-ecology}; \textit{Global and Planetary Change}; \textit{Marine Geology}; \textit{Quaternary Science Reviews}; \textit{Microorganisms}; \textit{Biology}; \textit{Diversity}; \textit{Sustainability}; \textit{Water}; \textit{Journal of Plankton Research}; \textit{Bulletin de la Soci\'{e}t\'{e} G\'{e}ologique de France}; \textit{Comptes Rendus Palevol}; \textit{Revue de Micropal\'{e}ontologie} and \textit{Acta Palaeontologica Romaniae}. \\
\end{longtable}

\section{WORKSHOPS}
% \subsection{As organizer}
% \begin{longtable}{@{}p{0.1\linewidth} P{0.85\linewidth}}
% \end{longtable}
% \subsection{As participant}
\begin{longtable}{@{}p{0.1\linewidth} P{0.85\linewidth}@{}}
2021--22 & \href{https://www.paleosynthesis.nat.fau.de/index.php/biodeeptime/}{`BioDeepTIME: rhythms, aberrations, and drivers of ecological turnover from daily to million-year timescales'} PaleoSynthesis Workshop.\\
2018 & \href{https://github.com/macroecology/BioExtremes}{BioExtreme hackathon} at MfN.\\
2017 & \href{https://abcd.biowikifarm.net/wiki/Events:WorkshopEFG2017}{`Access to Geosciences: sharing and publishing data related to paleontological, mineralogical, and petrological objects using a common data standard'} workshop at MfN.\\
2010 & \href{http://fossilworks.org/?page=workshop}{`Paleobiology Database Intensive Workshop'} at Macquarie University, Sydney, Australia.\\
\end{longtable}

\section{PUBLIC OUTREACH}
\begin{longtable}{@{}p{0.1\linewidth} P{0.85\linewidth}@{}}
2021 & \href{https://github.com/plannapus/raup_model}{Interactive display} for the ``Biominerale -- das Geheimnis der Schale'' exhibition at the Museum f\"{u}r Naturkunde Magdeburg, Germany.\\
%& Asatryan G., Lazarus D., Harbott M., Todorovic S., Kaplan J. O., Lee C. E., Parmesan C., Renaudie J., Thomas H., Wu H., Richards C. L. \href{}{How do plants, animals and microbes interact with climate and respond to climate change?} \textit{Frontiers for Young Minds}.\\
2019 & Guest of the \href{https://www.museumfuernaturkunde.berlin/de/museum/veranstaltungen/aktionstag-zum-globalen-klimastreik-0}{Museum Salon} in the context of the ``Fourth Global Day of Climate Action'' event for `Fridays For Future' at the MfN.\\
 & Public talk on `Antarktisches Mikroplankton und vergangene Klimawandel' for the Fachgruppe Pal\"{a}ontologie at the Museum f\"{u}r Naturkunde Magdeburg, Germany.\\
 & `The ocean’s plankton and climate change; how to see the future from the bottom of the ocean' booth at the MfN during the \href{http://www.langenachtderwissenschaften.de/}{Lange Nacht der Wissenschaften} and the \href{http://www.lange-nacht-der-museen.de/}{Lange Nacht der Museen}.\\
2014 & `Tiefenzeit Geschichten der Zukunft: von Plankton, Muscheln und Klimawandel' booth at the MfN during the \href{http://www.langenachtderwissenschaften.de/}{Lange Nacht der Wissenschaften}.\\
2011 & Guided tour of the MfN Micropaleontology Collection during the \href{http://www.lange-nacht-der-museen.de/}{Lange Nacht der Museen}.\\
\end{longtable}

\section{GRANTS \& AWARDS}
\subsection{Grants}
\begin{longtable}{@{}p{0.1\linewidth} P{0.85\linewidth}@{}}
2018 & DAAD `Make Our Planet Great Again--German Research Initiative' grant 57429681 (PI:Asatryan).\\
2015 & DFG Grant RE3470/3--1: `Eigene Stelle' grant in the Priority Program 527 (IODP).\\
\end{longtable}
\subsection{Awards}
\begin{longtable}{@{}p{0.1\linewidth} P{0.85\linewidth}@{}}
2012 & TMS Student Prize for best Poster at InterRad 13th meeting in Cadiz, Spain.\\
\end{longtable}

\section[MEDIA COVERAGE]{MEDIA COVERAGE \textnormal{\footnotesize{ of projects I'm involved with (selection)}}}
\begin{longtable}{@{}p{0.1\linewidth} P{0.85\linewidth}@{}}
2020 & \emph{NevadaToday}: \href{https://www.unr.edu/nevada-today/news/2020/marine-plankton}{Marine Plankton face threat of extinction as planet warms}.\\
 & \emph{IEEE Spectrum}: \href{https://spectrum.ieee.org/computing/software/ambitious-data-project-aims-to-organize-the-worlds-geoscientific-records}{Ambitious data project aims to organize the world’s geoscientific records}.\\
2019 & \emph{BBC}: \href{https://www.bbc.com/reel/video/p07bvx51/the-time-machines-unlocking-antarctica-s-past}{The `time machines' unlocking Antarctica's past}.\\
  & \emph{Science}: \href{http://dx.doi.org/10.1126/science.aax7040}{Newly drilled sediment cores could reveal how fast the Antarctic ice sheet will melt}.\\
2018 & \emph{DAAD Aktuell}: \href{https://www.daad.de/der-daad/daad-aktuell/de/66800-make-our-planet-great-again-german-research-initiative-forschung-fuer-die-zukunft-der-erde/}{``Make Our Planet Great Again -- German Research Initiative'': Forschung für die Zukunft der Erde}.\\
\end{longtable}

\section{WORK EXPERIENCE}
\begin{longtable}{@{}p{0.2\linewidth} P{0.75\linewidth}@{}}
09/2018--06/2022 & Postdoctoral researcher (DAAD MOPGA-GRI) at the MfN.\\
11/2015--05/2017 & Postdoctoral researcher (DFG) at the MfN.\\
03/2014--06/2015 & Postdoctoral researcher (Earthtime-EU) at the MfN.\\
05--07/2013 & Programmer at the MfN for the Neptune Database web interface.\\
11/2008--09/2012 & Scientific assistant at the MfN.\\
07--08/2008 & Bartender at the brasserie `Le Jardin Gourmand', Castelnaud-la-Chapelle.\\
10/2007--07/2008 & Library assistant at the Biblioth\`{e}que Centrale of Universit\'{e} Paris VII.\\
07--08/2003--2007 & Bartender at the restaurant `Le Bouffon', Sarlat-la-Can\'{e}da.\\
\end{longtable}

\section{OTHER}
\begin{longtable}{@{}p{0.94\linewidth} p{0.01\linewidth}@{}}
Fluent in French (native) and English, intermediate level in Latin and German. & \\
Advanced programming skills in R, python (including Django and wxPython), postgreSQL (and other SQL derivatives) and \LaTeX. Familiar with Unix shell and Apache Webserver. & \\
\end{longtable}

\vfill

{\let\thefootnote\relax\footnote{Last updated on \today\\
\theendnotes}}

\end{document}
