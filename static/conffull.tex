\documentclass[11pt, a4paper]{article}
\usepackage[margin=0.65cm]{geometry}
\usepackage{array}
\usepackage{longtable,needspace}
\usepackage{titlesec}
\usepackage{fontspec}
\usepackage{fancyhdr}
\usepackage{hyperref}
\usepackage{xcolor}
\usepackage{endnotes}
\usepackage{hyperendnotes}

\definecolor{grey}{RGB}{153,153,153}
\hypersetup{
    colorlinks,
    linkcolor=grey,
    urlcolor=grey,
    pdftitle={Complete conference participation},
    pdfauthor={Johan Renaudie},
    bookmarksopen=true,
    bookmarksopenlevel=1,
    pdfstartview=Fit
}
\setmainfont{Trade Gothic LT Std}
\titleformat{\section}{\large \bfseries}{\thesection}{1em}{}
\titleformat*{\subsection}{\normalsize \bfseries}
\setcounter{secnumdepth}{0}
\titlespacing{\section}{0pt}{3.25ex plus 1ex minus .2ex}{0pt}
\titlespacing{\subsection}{0pt}{0pt}{0pt}
\setlength{\LTpre}{0pt}
\newcommand{\dohang}{\hangindent1em\hangafter1 }
\newcolumntype{P}[1]{>{\everypar{\dohang}\dohang\raggedright\arraybackslash} p{#1}}
\pagestyle{fancy}
\fancyfoot{}
\fancyhead{}
\renewcommand{\headrulewidth}{0pt}
\renewcommand{\footnoterule}{\hrule width \linewidth height 1pt}
\makeatletter
\newcommand\fnoteref[1]{\protected@xdef\@theenmark{\ref{#1}}\@endnotemark}
\makeatother
\renewcommand*{\enoteheading}{}

\begin{document}
\subsection[Talks]{Talks \textnormal{\footnotesize{(\underline{speaker})}}}
\begin{longtable}{@{}p{0.1\linewidth} P{0.85\linewidth}@{}}
2021 & Crossing the Paleontological-Ecological Gap symposium in Berlin, Germany: \underline{Renaudie J.}, Özen V., Rodrigues de Faria G., Trubovitz S., Lazarus, D.B., Climatic range of modern fossilizable phytoplankton; \underline{Hunter J.}, Özen V., Rodrigues de Faria G., Renaudie J., Lazarus D., Southern Ocean Diatom Size Dynamics and the end-Eocene Paleoproductivity; \underline{Özen V.}, Rodrigues de Faria G., Renaudie J., Lazarus D., Evolutionary dynamics of the Southern Ocean diatoms across the Eocene-Oligocene transition; \underline{Rillo M.}, Smith J., Hull P., Finnegan S., the BioDeepTime Working Group, Integrating community turnover from modern and fossil data.\\
& Goldschmidt2021 in Lyon, France: \underline{Fontorbe G.}, Frings P.J., Renaudie J., Cao Z., Zhang Z., Frank M., Radiolarian species-specific fractionation: insights from a Miocene sediment core.\\
2020 & Progressive Palaeontology 2020 Online: \underline{Woodhouse A.}, Fenton I., Aze T., Lazarus D., Renaudie J., Young J., Saupe E., Triton: a new extension of the Neptune Database.\\
 & AGU Fall Meeting 2020 Online Everywhere: \underline{Woodhouse A. D.}, Fenton I., Jackson S., Saupe E., Dunhill A., Renaudie J., Lazarus D. B., Sexton P. F., Young J. R., Pearson P. N., Wignall P., Aze T., Can the Cenozoic marine paleontological record be used to predict multi-scaled extinction susceptibility? \\
2019 & 3rd International Congress on Stratigraphy in Milan, Italy: \underline{Renaudie J.}, Lazarus D., Diver P., \href{http://plannapus.github.io/data/20190723Potsdam.pdf}{NSB, a Big Data tool for chronostratigraphic syntheses of the deep-sea sediment record}.\\
 & \href{https://escholarship.org/uc/item/6r18f8wn}{North American Paleontological Conference (NAPC)} in Riverside, USA: \underline{Lazarus D.}, Renaudie J., Asatryan G. Diversity dynamics and climate change in Cenozoic marine siliceous plankton; \underline{Lazarus D.}, Renaudie J., Young J., Diver P., NSB and Mikrotax: Databases and software tools for fossil and living plankton research; \underline{Trubovitz S.}, Lazarus D., Renaudie J., Noble P., Tropical and polar plankton demonstrate contrasting sensitivities to climate change throughout the Late Neogene.\\
 & 3rd International Conference of Continental Ichnology (ICCI) in Halle, Germany: \underline{Jansen M.}, Buchwitz M., Renaudie J., Voigt S., Reconstruction of an Ancestral Amniote Trackmaker based on Trackway Data, Track -- Trackmaker Correlation and Phylogeny.\\
& Biodiversity\_Next in Leiden, Netherlands: \underline{Lazarus D.}, Renaudie J., \href{http://dx.doi.org/10.3897/biss.3.37066}{Paleobiodiversity and Earth Science Environmental Data}.\\
 & TMS Annual Meeting in Nottingham, UK: \underline{Young J.R.}, Lazarus D.B., Renaudie J., Bown P.R., Wade B.S., Huber B.T., Can we extract biostratigraphically useful data from large-scale occurrence-databases such as Neptune? Insights from development of the Mikrotax system.\\
  & AGU Fall Meeting 2019 in San Francisco, USA: \underline{Trubovitz S.}, Lazarus D.B., Renaudie J., Noble P.J., Neogene Radiolarian Climate Sensitivity and its Implications for Ocean Ecosystems and Geochemical Cycling; \underline{Wellner J.}, Gohl K., Klaus A. and the Expedition 379 Science Party\endnote{\label{exp379scientists}Gohl K., Wellner J., Klaus A., Bauersachs T., Bohaty S.M., Courtillat M., Cowan E.A., Esteves M.S.R., De Lira Mota M.A., Fegyveresi J.M., Frederichs T.W., Gao L., Halberstadt A.R., Hillenbrand C.-D., Horikawa K., Iwai M., Kim J.-H., King T.M., Klages J.P., Passchier S., Penkrot M.L., Prebble J.G., Rahaman W., Reinardy B.T.I., Renaudie J., Robinson D.E., Scherer R.P., Siddoway C.S., Wu L., Yamane M.}, West Antarctic Ice Sheet and Ocean Dynamics in the Outer Amundsen Sea: Initial Results from IODP Expedition 379.\\
2018 & GEOBONN 2018 `Living Earth' in Bonn, Germany: \underline{Jansen M.}, Buchwitz M., Renaudie J., Voigt S., Reconstruction of an ancestral amniote trackmaker based on trackway data, track-trackmaker correlation and phylogeny.\\
2017 & \href{http://interrad2017.random-walk.org/wp-content/uploads/2017/10/Abstracts_InterRadXV_171016b.pdf}{InterRad 15th meeting} in Niigata, Japan: \underline{Renaudie J.}, Fontorbe G., Lazarus D., Salzmann S., Frings P., Conley D., Testing the vital effect on silicon isotope measurements in Late Eocene Pacific radiolarians.\\
2016 & Lyell Meeting of the Geological Society in London, UK: \underline{Lazarus D.}, Renaudie J., Diver P., The NSB (Neptune) Database : current status and future development.\\
2015 & InterRad 14th meeting in Antalya, Turkey: \underline{Renaudie J.}, Lazarus D., Diatoms, radiolarians and the Cenozoic Si and C cycles ; \underline{Lazarus D.}, Renaudie J., Reconstructing radiolarian diversity: what we don't know; and a new analysis of the Cenozoic.\\
2014 & EGU general meeting in Vienna, Austria: \underline{Renaudie J.}, Lazarus D., A quantitative review of Cenozoic diatom deposition history.\\
2013 & TMS Silicofossil Group meeting in Cambridge, UK: \underline{Renaudie J.}, Lazarus D., Cenozoic history of marine diatom deposition: a quantitative review.\\
 & Deutscher Museumsbund Fachgruppe Dokumentation Herbsttagung in Berlin-Dahlem, Germany: \underline{Renaudie J.} (on behalf of Lazarus D.), Taxonomic backbone databases for fossil and living marine microplankton.\\
2012 & InterRad 13th meeting in Cadiz, Spain: \underline{Renaudie J.}, Lazarus D., On the accuracy of paleodiversity reconstructions: a case study using Antarctic Neogene radiolarians.\\
 & Jubil\"{a}umstagung der Pal\"{a}ontologische Gesellschaft in Berlin, Germany: \underline{Renaudie J.}, Lazarus D., Macroevolutionary patterns in Antarctic Neogene radiolarians.\\
 & TMS annual meeting in Nottingham, UK: \underline{Renaudie J.}, Lazarus D., Toward an high-resolution stratigraphy for Antarctic Neogene radiolarians.\\
2011 & TMS Silicofossil Group meeting in Lille, France: \underline{Renaudie J.}, Lazarus D., Macro-evolutionary patterns in Antarctic Neogene radiolarians.\\
2007 & Congr\`{e}s de l'Association Fran\c{c}aise de Pal\'{e}ontologie in Dignes-les-Bains, France: \underline{Renaudie J.}, Danelian T., Saint-Martin S., Eocene diatoms in the tropical Atlantic (ODP Leg 207) and climate changes.\\
\end{longtable}

\subsection[Posters]{Posters \textnormal{\footnotesize{(\underline{presenting})}}}
\begin{longtable}{@{}p{0.1\linewidth} P{0.85\linewidth}@{}}
2020 & Ocean Sciences meeting in San Diego, USA: \underline{Trubovitz S.}, Lazarus D., Renaudie J., Noble P., Radiolarians exhibit a threshold response to climate change during the late Neogene.\\
 & TMS General Meeting online: \underline{\"{O}zen V.}, Rodrigues de Faria G., Renaudie J., Lazarus D., Southern Ocean diatom diversity at the Eocene-Oligocene transition; \underline{Rodrigues de Faria G.}, Renaudie J., Struck U., Lazarus D., Southern Ocean productivity across the Eocene-Oligocene boundary.\\
2019 & EGU general meeting in Vienna, Austria: \underline{Asatryan G.}, Lazarus D., Renaudie J., The preliminary studies of plankton in the framework of the project ``Paleogene Polar Plankton and Paleoproductivity''.\\
& \href{https://escholarship.org/uc/item/6r18f8wn}{NAPC} in Riverside, USA: \underline{Trubovitz S.}, Lazarus D., Renaudie J., Noble P., New census of radiolarian communities in the Eastern Equatorial Pacific reveals unprecedented biodiversity throughout the Late Neogene.\\
& Society of Vertebrate Paleontologists (SVP) Annual Meeting in Brisbane, Australia: \underline{Jansen M.}, Buchwitz M., Renaudie J., Voigt S., Reconstruction of an ancestral amniote trackmaker based on trackway data, trackmaker correlation and phylogeny.\\
& Paläontologisches Gesellschaft General Meeting in Munich, Germany: \underline{\"{O}zen V.}, Rodrigues de Faria G., Asatryan G., Renaudie J., Lazarus D., Investigating the role of Southern Ocean phytoplankton in the end Eocene climatic events.\\
2018 & TMS Annual meeting in Leeds, UK: Asatryan G., Lazarus D., \underline{Renaudie J.}, Paleogene polar phytoplankton and oceanic carbon sequestration; \underline{Renaudie J.}, Drews E.-L., Böhne S., The Paleocene fossil record of marine planktonic diatoms in deep-sea sediments; \underline{Renaudie J.}, Gray, R., Lazarus D., Testing the accuracy of the MobileNet convolutional neural network to identify closely-related radiolarian species based on a sparse dataset.\\
& 25th International Diatom Symposium in Berlin, Germany: \underline{Renaudie J.}, Lazarus D., Macroevolutionary patterns in Cenozoic marine diatoms from deep-sea sediments, and their relationship with climate and marine geochemical cycles.\\
2017 & Kolloquium der DFG-Schwerpunkte ICDP/IODP in Braunschweig, Germany: \underline{Renaudie J.}, Fontorbe G., Drews E.-F., Böhne S., Lazarus D., Constraining the history of the Cenozoic marine silicon cycle with siliceous microfossils.\\
& Evolution 2017 in Portland, Oregon: Renaudie J., \underline{Lazarus D.}, Diver P., The NSB (Neptune) marine microfossil occurrences database.\\
2016 & Kolloquium der DFG-Schwerpunkte ICDP/IODP in Heidelberg, Germany: \underline{Renaudie J.}, Siliceous microfossils and the Cenozoic marine carbon and silicon cycles ; Renaudie J., Diver P., \underline{Lazarus D.}, NSB and ADP: a new, expanded and improved software system for marine planktonic microfossil and geochronologic data; \underline{Wiese R.}, Renaudie J., Lazarus D., Can genera be used as proxies for species in studies of biodiversity-climate sensitivity? A test using Cenozoic marine diatoms.\\
 & International Conference on Paleoceanography in Utrecht, Netherlands: \underline{Renaudie J.}, Quantifying the Cenozoic marine diatom record; \underline{Renaudie J.}, Lazarus D., Diver P., Expanding the NSB database for paleoceanographical research.\\
2015 & Kolloquium der DFG-Schwerpunkte ICDP/IODP in Bonn, Germany: \underline{Renaudie J.}, Diatoms and the Cenozoic Si and C cycles.\\
 & GSA annual meeting in Baltimore, USA: Renaudie J., Diver P., \underline{Lazarus D.}, NSB: a new, expanded and improved database of marine planktonic microfossil data.\\
2014 & Kolloquium der DFG-Schwerpunkte ICDP/IODP in Erlangen, Germany: \underline{Renaudie J.}, Lazarus D.B., A quantitative review of Cenozoic diatom deposition history.\\
 & EGU general meeting in Vienna, Austria: \underline{Lazarus D.B.}, Renaudie J., Diversity history of Cenozoic marine siliceous plankton.\\
2013 & Kolloquium der DFG-Schwerpunkte ICDP/IODP in Freiberg, Germany: Renaudie J., \underline{Lazarus D.B.}, Toward an high-resolution stratigraphy for Antarctic Neogene radiolarians.\\
2012 & Kolloquium der DFG-Schwerpunkte ICDP/IODP in Kiel, Germany: \underline{Renaudie J.}, Lazarus D.B., Macroevolutionary patterns in Antarctic Neogene radiolarians.\\
 & InterRad 13th meeting in Cadiz, Spain: \underline{Renaudie J.}, Lazarus D.B., Advances in Antarctic Neogene radiolarian high-resolution stratigraphy; Lazarus D.B., \underline{Renaudie J.}, Taxonomic documentation of Southern Ocean Neogene radiolarians.\\
 & TMS annual meeting in Nottingham, UK: \underline{Renaudie J.}, Lazarus D.B., Macroevolutionary patterns in Antarctic Neogene radiolarians.\\
2011 & Kolloquium der DFG-Schwerpunkte ICDP/IODP in Münster, Germany: \underline{Renaudie J.}, Lazarus D.B., Paleodiversity reconstructions in Antarctic Neogene radiolarians\\
2010 & Kolloquium der DFG-Schwerpunkte ICDP/IODP in Frankfurt, Germany: Renaudie J., \underline{Lazarus D.B.}, Weinkauf M., Marine micropaleontology, round two: whole faunal surveys and their use in biostratigraphy and macroevolutionary research.\\
 & Third International Paleontological Congress in London, UK: Renaudie J., \underline{Lazarus D.B.}, Macroevolutionary patterns in Antarctic Neogene radiolarians.\\
2009 & Kolloquium der DFG-Schwerpunkte ICDP/IODP in Potsdam, Germany: Lazarus D.B., \underline{Renaudie J.}, Synthesis and analysis of Antarctic Neogene Radiolaria.\\
 & InterRad 12th meeting in Nanjing, China: Lazarus D.B., \underline{Renaudie J.}, Synthesis on Antarctic Neogene Radiolaria: preliminary results.\\
2007 & Paleontological Association 51st Annual Meeting in Uppsala, Sweden: Renaudie J., \underline{Danelian T.}, Saint-Martin S., The siliceous plankton response of the equatorial Atlantic to the Middle Eocene Climatic Optimum event (ODP Site 1260, Demerara Rise, off Surinam).\\
2006 & Paleontological Association 50th Annual Meeting in Sheffield, UK: \underline{Danelian T.}, Renaudie J., Blanc-Valleron M.-M., The radiolarian record of the equatorial Atlantic during the Paleocene/Eocene Thermal Maximum event.\\
\end{longtable}
\end{document}
